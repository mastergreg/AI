\documentclass[a4paper,10pt]{article} \usepackage{anysize}
\marginsize{2cm}{2cm}{1cm}{1cm}
%\textwidth 6.0in \textheight = 664pt
\usepackage{xltxtra}
\usepackage{xunicode} \usepackage{graphicx}
\usepackage{color} \usepackage{xgreek} \usepackage{fancyvrb}
\usepackage{minted}
\usepackage{listings}
\usepackage{enumitem} \usepackage{framed} \usepackage{relsize}
\usepackage{float} \setmainfont[Mapping=TeX-text]{DejaVu Sans}
\begin{document}

\include{title/title}

\section*{Υλοποίηση αλγορίθμου A*} \setcounter{section}{1}
Ο πηγαίος κώδικας της main.c που κληθήκαμε να γράψουμε ήταν ο εξής:

Στη συνέχεια δημιουργήσαμε το makefile για  τη μεταγλώττιση του προγράμματος
με τα εξής περιεχόμενα:


Τρέχοντας στο shell την εντολή make έχουμε την παρακάτω έξοδο

\section*{Απαντήσεις στις θεωρητικές ερωτήσεις}
Kakamaka
\end{document}
